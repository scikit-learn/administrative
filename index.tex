\documentclass[twoside,11pt]{article} 
\usepackage{jmlr2e} 

\jmlrheading{1}{2000}{1-48}{4/00}{10/00}{Marina Meil\u{a} and Michael I. Jordan}

% Short headings should be running head and authors last names

%%%%%%%%%%%%%%%%%%%%%%%%%%%%%%%%%%%%%%%%%%%%%%%%%
%%%%%%%                                   %%%%%%%
%%%%%%% GAEL: YEAH THIS IS STILL HORRIBLE %%%%%%%
%%%%%%%                                   %%%%%%%
%%%%%%%%%%%%%%%%%%%%%%%%%%%%%%%%%%%%%%%%%%%%%%%%%


%% THERE IS STILL NO MESSAGE WHY WE ARE BETTER !!!


\ShortHeadings{Scikit-learn: machine learning in Python}{Meil\u{a} and Jordan}
\firstpageno{1}


\begin{document}

\title{Scikit-learn: machine learning in Python}


\author{\name Marina Meil\u{a} \email mmp@stat.washington.edu \\
       \addr Department of Statistics\\
       University of Washington\\
       Seattle, WA 98195-4322, USA
       \AND
       \name Michael I.\ Jordan \email jordan@cs.berkeley.edu \\
       \addr Division of Computer Science and Department of Statistics\\
       University of California\\
       Berkeley, CA 94720-1776, USA}

\editor{?}


\maketitle

\begin{abstract}
scikits.learn is a Python module integrating classic machine learning
algorithms in the tightly-knit world of scientific Python packages
(numpy, scipy, matplotlib).
\end{abstract}

\keywords{Python, machine learning}



\section{Introduction}
In the past years python has experienced a huge increase as the
language of choice for scientific computing. It's interactive nature,
it's openness and the maturation of some of its numerical libraries
(numpy, scipy) have all contributed to its success.

Because of this, libraries that are too domain specific to be included
within the general-purpose scipy libraries are distributed in the form
of scikits. In this paper, we present one of such packages:
scikit-learn, a library for machine learning.

%% this is just the same lame thing everyone says
When building this library, we wanted it to be fast, simple ...



\begin{center}
%% \caption{Overview of existing machine learning libraries}

% it would be nicer with checkbox images

\begin{tabular}{c c c c c c c}
\hline\hline %inserts double horizontal lines 
implements & SKL & PyML & pybrain & PyMVPA &  MDP & shogun \\ [0.5ex]
\hline
svm        & Yes & ?   & ?       &  ?     & ?    & ? \\
\hline
\end{tabular}
\end{center}


\section{Technology}

We based our developement around the technologies numpy and
cython. Numpy arrays are the choice for representing data. It is easy
to convert/load numpy arrays from virtually any existing format: .mat,
csv, gzip, etc.

Cython is used to speedup some algorithms (CD, SGD) and to interface C
code (talk about libsvm).

\section{Code by implementation, not by inheritance}

To have a uniform API, we adopt simple conventions and limit to a
minimum the number of methods an object has to implement. However, we
do not enforce inheritance, any object implementing these methods is
an estimator.


Estimators with .fit(), .predict()


\section{Community}

Examples, contributions, github.





\end{document}
