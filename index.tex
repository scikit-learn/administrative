\documentclass[twoside,11pt]{article} 
\usepackage{jmlr2e} 
\usepackage{hyperref}
\jmlrheading{1}{2011}{1-48}{4/00}{10/00}{Pedregosa, Varoquaux et al.}

% Short headings should be running head and authors last names

%%%%%%%%%%%%%%%%%%%%%%%%%%%%%%%%%%%%%%%%%%%%%%%%%
%%%%%%%                                   %%%%%%%
%%%%%%% GAEL: YEAH THIS IS STILL HORRIBLE %%%%%%%
%%%%%%%                                   %%%%%%%
%%%%%%%%%%%%%%%%%%%%%%%%%%%%%%%%%%%%%%%%%%%%%%%%%



\ShortHeadings{Scikit-learn: machine learning in Python}{Pedregosa, Varoquaux et al.}
\firstpageno{1}


\begin{document}

\title{Scikit-learn: machine learning in Python}


\author{\name Fabian Pedregosa \email fabian.pedregosa@inria.fr \\
        \name Gael Varoquaux \email gael.varoquaux@normalesup.org  \\
        \name Alexandre Gramfort \email TODO (INRIA?)\\
        \name Vincent Michel  \email TODO \\
       \addr Department of Statistics\\
       University of Washington\\
       Seattle, WA 98195-4322, USA
       \AND
       \name Olivier Grisel \email jordan@cs.berkeley.edu \\
       \addr Division of Computer Science and Department of Statistics\\
       University of California\\
       Berkeley, CA 94720-1776, USA}


 %% David Cournapeau ?

\editor{?}


\maketitle

\begin{abstract}
This paper describes scikits.learn, a Python module integrating
various machine learning algorithms under a common interface. It
offers a wide range of methods such as Support Vector Machines, linear
models (l1 and l2), logistic regression, gaussian mixture models and
more.

We have succesfully used this software to real-world problems, some of
them can be seen in the web page.

scikits.learn is licensed under the simplified BSD license,
encouraging its use in both free and commercial settings. Soruce code
and documentation can be downloaded from
\url{http://scikit-learn.sourceforge.net}.

\end{abstract}

\keywords{Python, open source software, support vector machines,
  lasso, elastic net.}



\section{Introduction}
In the past years the Python programming language has experienced a
huge increase as the language of choice for scientific computing. It's
interactive nature, it's openness and the maturation of some of the
scientific libraries (numpy, scipy) have all contributed to its
success.

However, machine learning libraries were confined to specific domains
(PyMVPA) or to concrete methods (PyBrain), and while several machine
learning libraries and frameworks provide Python bindings, those
interfaces are usually less pythonic than those designed exclusively
for the Python language.

 %% better word for pythonic

This motivated us to develop scikits.learn, a module that implements a
broad range of machine learning algorithms under a Pythonic
interface. Most code is written in Python, 

The modular design of scikits.learn makes it easy to implement new
algorithms. The large number of algorithms aleady implemented allows
for easy comparison of accuracy and performance of various algorithms.


\begin{center}
%% \caption{Overview of existing machine learning libraries}

% it would be nicer with checkbox images

\begin{tabular}{c c c c c c c}
\hline\hline %inserts double horizontal lines 
implements & scikits.learn & PyML & pybrain & PyMVPA &  MDP & Shogun \\ [0.5ex]
\hline
SVM        & + & ?   & ?       &  ?     & ?    & + \\
Elastic Net & + & ?   & ?       &  ?     & ?    & - \\

\hline
\end{tabular}
\end{center}


\section{Technology}

Key technologies are numpy and cython. Numpy arrays are the choice for
representing data. Numpy is the universal format for representing
arrays, all methods accept dense or sparse numpy arrays on entry, and
it is also easy to convert and load numpy arrays from existing formats:
.mat, csv, gzip, etc.

The purpose of Cython is twofold. First, it allows to reach the
performance of traditional compiled languages as C while retaining
some of the benefits of Python, such as clear syntax and garbage
collection.

Second, Cython allows to easily access third-party libraries. libsvm
and liblinear have been efficiently binded using Cython [...]. Bench ?


\section{Code by implementation, not by inheritance}

To have a uniform API, we adopt simple conventions and limit to a
minimum the number of methods an object has to implement. In its
simples form, a classifier is an object that implements a \emph{fit}
and \emph{predict} method. Inheritance is not enforced.

Filters implement \emph{transform}. Other imple

\section{Community}

Examples, contributions, github.

Ohloh describes the code as ``Well-commented source code'' and ``Very
large, active development team''.

\section{Drawnbacks}
In contrast to the benefits of writing python code (dynamic,
readable), we also have drawnbacks. As most algorithms are implemented
in Python or Cython, reuse in other languages is difficult.


\end{document}
